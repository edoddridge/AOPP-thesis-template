% Geoff Vallis' command definition commands:
%\newcommand{\newcm}[2]{\newcommand*{#1}{\ensuremath{#2}}}
%\newcommand{\mycommand}[2]{\providecommand{#1}{} \renewcommand{#1}{#2}}

%define or redefine commands for the whole document
\renewcommand{\vec}{\bm}
\newcommand{\DDt}[1]{\frac{D #1}{Dt}}
\newcommand{\pd}[2]{\frac{\partial #1}{\partial #2}}
\newcommand{\dvg}[1]{\ensuremath{\nabla \cdot \vec{#1}}} % Divergence
\newcommand{\curl}[1]{\nabla \times \vec{#1}}
\newcommand{\oiint}{ \bigcirc \hspace{-0.60cm} \int \hspace{-0.40cm}\int }
\newcommand{\norm}[1]{\lVert#1\rVert}
\newcommand{\sumN}{\sum\limits^{N}_{n=1}}



% Formatting:
%   Bold  \mathbf shortcut.   (Just use \bm from package 'bm').

% Derivatives
\newcommand{\di}{\partial}
\newcommand{\pypx}[2]{\frac{\partial{#1}}{\partial{#2}}}    % Partial deriv, Numerator & Denom
\newcommand{\ppx}[1]{\frac{\partial}{\partial{#1}}}      % Partial Derivative Operator
\newcommand{\pypxx}[2]{\frac{\partial^2{#1}}{\partial{#2}^2}}
\newcommand{\ddx}[1]{\frac{\mathrm{d}}{\mathrm{d}{#1}}}   % d Derivative operator
\newcommand{\dydx}[2]{\frac{\mathrm{d}{#1}}{\mathrm{d}{#2}}} % d deriv, Numerator and Denominator
\newcommand{\dydxx}[2]{\frac{\mathrm{d}^2{#1}}{\mathrm{d}{#2}^2}}
\newcommand{\DyDx}[2]{\frac{\mathrm{D}{#1}}{\mathrm{D}{#2}}} % D deriv, Numerator and Denominator


% Bracketing
\newcommand{\paren}[1]{\left( #1 \right)}
\renewcommand{\brack}[1]{\left[ #1 \right]}
\newcommand{\curly}[1]{\left{ #1 \right}}
\newcommand{\abs}[1]{\lvert #1 \rvert}


% Bracketing fractions
\newcommand{\pfrac}[2]{\left( \frac{#1}{#2} \right) }
\newcommand{\bfrac}[2]{\left[ \frac{#1}{#2} \right] }


% Letters etc. 
\newcommand{\D}{\mathrm{D}}
\renewcommand{\d}{\mathrm{d}}
\newcommand{\e}{\mathrm{e}}                               % Non-slanted e, for exponential.
\newcommand{\gn}{\ensuremath{\gamma^n}\xspace}                    % \gamma^n shortcut.

% For integration:
\newcommand{\dt}{\, \mathrm{d}t}
\newcommand{\dx}{\, \mathrm{d}x}
\newcommand{\dy}{\, \mathrm{d}y}
\newcommand{\dz}{\, \mathrm{d}z}


% Degrees
\renewcommand{\deg}{\ensuremath{^{\circ}}\xspace}
\newcommand{\degc}{\ensuremath{^{\circ}\mathrm{C}}\xspace}
\newcommand{\degn}{\ensuremath{^{\circ}\mathrm{N}}\xspace}
\newcommand{\degs}{\ensuremath{^{\circ}\mathrm{S}}\xspace}
\newcommand{\dege}{\ensuremath{^{\circ}\mathrm{E}}\xspace}
\newcommand{\degw}{\ensuremath{^{\circ}\mathrm{W}}\xspace}



% References:
\newcommand{\figref}[1]{Fig.~\ref{#1}}
\newcommand{\Figref}[1]{Figure \ref{#1}}
\newcommand{\secref}[1]{Section \ref{#1}}
\newcommand{\chref}[1]{Chapter \ref{#1}}

% Citations:
%     Citet with apostrophe, e.g. Marshall's [1995] idea was...
\newcommand{\citetapos}[1]{\citeauthor{#1}'s \citeyearpar{#1}}

% Words:
\newcommand{\ie}{i.e.\xspace}
\newcommand{\eg}{e.g.\xspace}


% UNITS  (just use \usepackage{siunitx} )
%\newcommand{\s}{\si{s}}

%\newcommand{\s}{\,\text{s}\xspace} 
%\newcommand{\W}{\,\text{W}\xspace} 
%\mycommand{\m}{\,\text{m}\xspace}
%\newcommand{\cm}{\,\text{cm}\xspace}
%\newcommand{\mm}{\,\text{mm}\xspace}
%\newcommand{\km}{\,\text{km}\xspace}
%\newcommand{\kg}{\,\text{kg}\xspace}
%\newcm{\ps}{\,\text{s}{^{-1}}\xspace}
%\newcommand{\mps}{\ensuremath{\m\ps}\xspace}
%\newcommand{\gpk}{\ensuremath{\,\text{g\,kg}^{-1}}\xspace}
%\newcommand{\Kv}{\,\text{K}\xspace}
%\newcommand{\Sv}{\,\text{Sv}\xspace}
%\newcommand{\hpa}{\,\text{hPa}\xspace}
%\newcommand{\hPa}{\,\text{hPa}\xspace}





